%!TEX root=../main.tex

% Packages
%

\usepackage[utf8]{inputenc}
\usepackage[T1]{fontenc}
\usepackage{hyperref}
\usepackage{verbatim}
\usepackage{tikz}
\usetikzlibrary{positioning,calc}
\usepackage{xspace}
\usepackage{mathtools}
\usepackage{pifont}
\usepackage{etoolbox}
\usepackage[normalem]{ulem}
\usepackage{booktabs}
\usepackage{array}
\usepackage[capitalise,noabbrev]{cleveref}
\usepackage{cite}
\usepackage{multibib}
\usepackage{url}
\usepackage{algorithm}
\usepackage{algpseudocode}
\usepackage{paralist}
\usepackage{mathrsfs}
\usepackage{relsize}
\usepackage{stmaryrd}
\usepackage[textsize=small]{todonotes}
\usepackage{multirow}

\DeclareFontFamily{U}{mathx}{\hyphenchar\font45}
\DeclareFontShape{U}{mathx}{m}{n}{<-> mathx10}{}
\DeclareSymbolFont{mathx}{U}{mathx}{m}{n}
\DeclareMathAccent{\widebar}{0}{mathx}{"73}

%\let\vec\vec
\usepackage{amssymb,amsmath}
\let\vec\boldsymbol


\usepackage[
n, % or lambda
advantage,
operators,
sets,
adversary,
landau,
probability,
notions,
logic,
ff,
mm,
primitives,
events,
complexity,
oracles,
asymptotics,
keys
]{cryptocode}

\newtoggle{notes}
\toggletrue{notes} % set to false to remove colored notes from the paper

%--------------------------------------------------------
% Editorial
%--------------------------------------------------------

\newcommand{\redunderline}[1]{\textcolor{red}{\underline{\textcolor{black}{#1}}}} 
\newcommand{\TBW}{\textcolor{blue}{\textbf{To Be Written...}}}
\newcommand{\Note}[1]{\textcolor{magenta}{ $\langle \! \langle$ #1 $\rangle \! \rangle$}}
\newcommand{\todonote}[1]{\todo[inline]{MyName: #1}}

\DeclareRobustCommand{\ahmets}[2]  {{\color{blue} $[$ \textsf{Ahmet #1:} #2 $]$ }}
\DeclareRobustCommand{\michals}[2]  {{\color{magenta} $[$ \textsf{Michal #1:} #2 $]$ }}

%--------------------------------------------------------
% General notations
%--------------------------------------------------------


\usepackage{xspace}
\newcommand{\ie}{i.e.\@\xspace}
\newcommand{\sA}{{\mathcal A}}
\newcommand{\sB}{{\mathcal B}}
\newcommand{\sC}{{\mathcal C}}
\newcommand{\sD}{{\mathcal D}}
\newcommand{\sE}{{\mathcal E}}
\newcommand{\sF}{{\mathcal F}}
\newcommand{\sG}{{\mathcal G}}
\newcommand{\sH}{{\mathcal H}}
\newcommand{\sI}{{\mathcal I}}
\newcommand{\sJ}{{\mathcal J}}
\newcommand{\sK}{{\mathcal K}}
\newcommand{\sL}{{\mathcal L}}
\newcommand{\sM}{{\mathcal M}}
\newcommand{\sN}{{\mathcal N}}
\newcommand{\sO}{{\mathcal O}}
\newcommand{\sP}{{\mathcal P}}
\newcommand{\sQ}{{\mathcal Q}}
\newcommand{\sR}{{\mathcal R}}
\newcommand{\sS}{{\mathcal S}}
\newcommand{\sT}{{\mathcal T}}
\newcommand{\sU}{{\mathcal U}}
\newcommand{\sV}{{\mathcal V}}
\newcommand{\sW}{{\mathcal W}}
\newcommand{\sX}{{\mathcal X}}
\newcommand{\sY}{{\mathcal Y}}
\newcommand{\sZ}{{\mathcal Z}}

% Sonderbuchstaben mit Doppellinie
\renewcommand{\pcsetstyle}[1]{\ensuremath{\mathbb{#1}}}

\renewcommand{\probname}{Pr}
\renewcommand{\pcadvstyle}[1]{\ensuremath{\mathcal{#1}}}

\renewcommand{\pcadvantagename}{\mathsf{Adv}}
\renewcommand{\pcadvantagesuperstyle}[1]{\mathrm{\MakeLowercase{#1}}}
\renewcommand{\pcadvantagesubstyle}[1]{#1}

\renewcommand{\pckeystyle}[1]{\ensuremath{\mathsf{#1}}}


%--------------------------------------------------------
% Standard proba, games, proofs, sampling, distributions
%--------------------------------------------------------

\newcommand{\Good}{\ensuremath{\mathsf{Good}}\xspace}
\newcommand{\Bad}{\ensuremath{\mathsf{Bad}}\xspace}
\newcommand{\equivStat}{\ensuremath{\overset{\mathsf{stat}}{\equiv}}\xspace}
\newcommand{\equivComp}{\ensuremath{\overset{\mathsf{comp}}{\equiv}}\xspace}
\newcommand{\view}{\ensuremath{\textsc{View}}\xspace}

\newcommand{\Hyb}{\ensuremath{\mathsf{Hyb}}\xspace}
\newcommand{\Exp}{\ensuremath{\mathsf{Exp}}\xspace}
\newcommand{\myGame}{\ensuremath{\mathsf{Game}}\xspace}
\newcommand{\Event}{\ensuremath{\mathsf{E}}\xspace}
\newcommand{\Span}{\ensuremath{\mathsf{Span}}}

\newcommand{\probb}[2]{{\Pr}_{#1}\left[\,{#2}\,\right]}
\newcommand{\Dx}{\mathcal{D}}
\newcommand{\Hx}{\mathcal{H}}
\newcommand{\Sx}{\mathcal{S}}
\newcommand{\Lx}{\mathcal{L}}
\newcommand{\Dist}{\mathcal{D}}
\newcommand{\Expect}{\ensuremath{\mathbb{E}}\xspace}
\newcommand{\Sample}{\ensuremath{\mathsf{Sample}}\xspace}
\newcommand{\Sim}{\ensuremath{\mathsf{Sim}}\xspace}


%--------------------------------------------------------
% Hard problems
%--------------------------------------------------------

\newcommand{\hardprobfont}[1]{\texorpdfstring{\ensuremath{\textsf{#1}}}{#1}}
\newcommand{\kLIN}{\ensuremath{k\text{-}\mathsf{Lin}}\xspace}
\newcommand{\SEDL}{\hardprobfont{SEDL}\xspace}
\newcommand{\DL}{\hardprobfont{DL}\xspace}
\newcommand{\DDH}{\hardprobfont{DDH}\xspace}
\newcommand{\kerDH}{\hardprobfont{kerDH}\xspace}
\newcommand{\kernel}{\mathsf{ker}\xspace}
\newcommand{\DHP}{\hardprobfont{DH}\xspace}
\newcommand{\DLin}{\hardprobfont{DLin}\xspace}
\newcommand{\XDH}{\hardprobfont{XDH}\xspace}
\newcommand{\CDH}{\hardprobfont{CDH}\xspace}
\newcommand{\LWE}{\hardprobfont{LWE}\xspace}
\newcommand{\SXDH}{\hardprobfont{SXDH}\xspace}
\newcommand{\DCR}{\hardprobfont{DCR}\xspace}
\newcommand{\dlog}{\ensuremath{\mathsf{dlog}}\xspace}

%--------------------------------------------------------
% Various
%--------------------------------------------------------

\newcommand{\map}{\ensuremath{\mathsf{map}}\xspace}
\newcommand{\mode}{\mathsf{mode}}
\newcommand{\token}{\ensuremath{\mathsf{token}}\xspace}
\newcommand{\seed}{\ensuremath{\mathsf{seed}}\xspace}
\newcommand{\inp}{\ensuremath{\mathsf{in}}\xspace}
\newcommand{\outp}{\ensuremath{\mathsf{out}}\xspace}
\newcommand{\circuit}{\ensuremath{\mathcal{C}}\xspace}
\newcommand{\size}[1]{\ensuremath{\left\vert #1 \right\vert}\xspace}
\newcommand{\myand}{\ensuremath{\mathsf{and}}\xspace}
\newcommand{\myxor}{\ensuremath{\mathsf{xor}}\xspace}

\newcommand{\mynot}{\ensuremath{\mathsf{not}}\xspace}
\newcommand{\Input}{\ensuremath{\mathsf{Input}}\xspace}


\newcommand{\hsha}{{\textrm{HMAC-SHA256}}}
\createpseudocodeblock {pcb}{center, boxed}{}{}{}

\newcommand{\samplePoly}{\pcalgostyle{Samp}}

\newcommand{\smallset}[1]{\{#1\}}