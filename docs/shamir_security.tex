%
% Parameters
%

\ifdefined\fullversion
\else
\def\fullversion{1}    % 0 = conference version; 1 = full version
\fi

\ifdefined\cameraversion
\else
\def\cameraversion{0}    % 0 = long version; 1 = proceedings version
\fi

\def\showoverflow{1}   % 1 = show overflows
\def\allow{1}      % 0 = remove todo command
\def\anonymous{1}      % 1 = anonymous

%
% Document class 
%

\documentclass[envcountsame,runningheads,notitlepage]{llncs}


\ifnum\fullversion=1
\usepackage[a4paper, margin=1.5in]{geometry}
\setlength{\marginparwidth}{2.5cm}
\fi



%
% Custom header
%

%!TEX root=../main.tex

% Packages
%

\usepackage[utf8]{inputenc}
\usepackage[T1]{fontenc}
\usepackage{hyperref}
\usepackage{verbatim}
\usepackage{tikz}
\usetikzlibrary{positioning,calc}
\usepackage{xspace}
\usepackage{mathtools}
\usepackage{pifont}
\usepackage{etoolbox}
\usepackage[normalem]{ulem}
\usepackage{booktabs}
\usepackage{array}
\usepackage[capitalise,noabbrev]{cleveref}
\usepackage{cite}
\usepackage{multibib}
\usepackage{url}
\usepackage{algorithm}
\usepackage{algpseudocode}
\usepackage{paralist}
\usepackage{mathrsfs}
\usepackage{relsize}
\usepackage{stmaryrd}
\usepackage[textsize=small]{todonotes}
\usepackage{multirow}

\DeclareFontFamily{U}{mathx}{\hyphenchar\font45}
\DeclareFontShape{U}{mathx}{m}{n}{<-> mathx10}{}
\DeclareSymbolFont{mathx}{U}{mathx}{m}{n}
\DeclareMathAccent{\widebar}{0}{mathx}{"73}

%\let\vec\vec
\usepackage{amssymb,amsmath}
\let\vec\boldsymbol


\usepackage[
n, % or lambda
advantage,
operators,
sets,
adversary,
landau,
probability,
notions,
logic,
ff,
mm,
primitives,
events,
complexity,
oracles,
asymptotics,
keys
]{cryptocode}

\newtoggle{notes}
\toggletrue{notes} % set to false to remove colored notes from the paper

%--------------------------------------------------------
% Editorial
%--------------------------------------------------------

\newcommand{\redunderline}[1]{\textcolor{red}{\underline{\textcolor{black}{#1}}}} 
\newcommand{\TBW}{\textcolor{blue}{\textbf{To Be Written...}}}
\newcommand{\Note}[1]{\textcolor{magenta}{ $\langle \! \langle$ #1 $\rangle \! \rangle$}}
\newcommand{\todonote}[1]{\todo[inline]{MyName: #1}}

\DeclareRobustCommand{\ahmets}[2]  {{\color{blue} $[$ \textsf{Ahmet #1:} #2 $]$ }}
\DeclareRobustCommand{\michals}[2]  {{\color{magenta} $[$ \textsf{Michal #1:} #2 $]$ }}

%--------------------------------------------------------
% General notations
%--------------------------------------------------------


\usepackage{xspace}
\newcommand{\ie}{i.e.\@\xspace}
\newcommand{\sA}{{\mathcal A}}
\newcommand{\sB}{{\mathcal B}}
\newcommand{\sC}{{\mathcal C}}
\newcommand{\sD}{{\mathcal D}}
\newcommand{\sE}{{\mathcal E}}
\newcommand{\sF}{{\mathcal F}}
\newcommand{\sG}{{\mathcal G}}
\newcommand{\sH}{{\mathcal H}}
\newcommand{\sI}{{\mathcal I}}
\newcommand{\sJ}{{\mathcal J}}
\newcommand{\sK}{{\mathcal K}}
\newcommand{\sL}{{\mathcal L}}
\newcommand{\sM}{{\mathcal M}}
\newcommand{\sN}{{\mathcal N}}
\newcommand{\sO}{{\mathcal O}}
\newcommand{\sP}{{\mathcal P}}
\newcommand{\sQ}{{\mathcal Q}}
\newcommand{\sR}{{\mathcal R}}
\newcommand{\sS}{{\mathcal S}}
\newcommand{\sT}{{\mathcal T}}
\newcommand{\sU}{{\mathcal U}}
\newcommand{\sV}{{\mathcal V}}
\newcommand{\sW}{{\mathcal W}}
\newcommand{\sX}{{\mathcal X}}
\newcommand{\sY}{{\mathcal Y}}
\newcommand{\sZ}{{\mathcal Z}}

% Sonderbuchstaben mit Doppellinie
\renewcommand{\pcsetstyle}[1]{\ensuremath{\mathbb{#1}}}

\renewcommand{\probname}{Pr}
\renewcommand{\pcadvstyle}[1]{\ensuremath{\mathcal{#1}}}

\renewcommand{\pcadvantagename}{\mathsf{Adv}}
\renewcommand{\pcadvantagesuperstyle}[1]{\mathrm{\MakeLowercase{#1}}}
\renewcommand{\pcadvantagesubstyle}[1]{#1}

\renewcommand{\pckeystyle}[1]{\ensuremath{\mathsf{#1}}}


%--------------------------------------------------------
% Standard proba, games, proofs, sampling, distributions
%--------------------------------------------------------

\newcommand{\Good}{\ensuremath{\mathsf{Good}}\xspace}
\newcommand{\Bad}{\ensuremath{\mathsf{Bad}}\xspace}
\newcommand{\equivStat}{\ensuremath{\overset{\mathsf{stat}}{\equiv}}\xspace}
\newcommand{\equivComp}{\ensuremath{\overset{\mathsf{comp}}{\equiv}}\xspace}
\newcommand{\view}{\ensuremath{\textsc{View}}\xspace}

\newcommand{\Hyb}{\ensuremath{\mathsf{Hyb}}\xspace}
\newcommand{\Exp}{\ensuremath{\mathsf{Exp}}\xspace}
\newcommand{\myGame}{\ensuremath{\mathsf{Game}}\xspace}
\newcommand{\Event}{\ensuremath{\mathsf{E}}\xspace}
\newcommand{\Span}{\ensuremath{\mathsf{Span}}}

\newcommand{\probb}[2]{{\Pr}_{#1}\left[\,{#2}\,\right]}
\newcommand{\Dx}{\mathcal{D}}
\newcommand{\Hx}{\mathcal{H}}
\newcommand{\Sx}{\mathcal{S}}
\newcommand{\Lx}{\mathcal{L}}
\newcommand{\Dist}{\mathcal{D}}
\newcommand{\Expect}{\ensuremath{\mathbb{E}}\xspace}
\newcommand{\Sample}{\ensuremath{\mathsf{Sample}}\xspace}
\newcommand{\Sim}{\ensuremath{\mathsf{Sim}}\xspace}


%--------------------------------------------------------
% Hard problems
%--------------------------------------------------------

\newcommand{\hardprobfont}[1]{\texorpdfstring{\ensuremath{\textsf{#1}}}{#1}}
\newcommand{\kLIN}{\ensuremath{k\text{-}\mathsf{Lin}}\xspace}
\newcommand{\SEDL}{\hardprobfont{SEDL}\xspace}
\newcommand{\DL}{\hardprobfont{DL}\xspace}
\newcommand{\DDH}{\hardprobfont{DDH}\xspace}
\newcommand{\kerDH}{\hardprobfont{kerDH}\xspace}
\newcommand{\kernel}{\mathsf{ker}\xspace}
\newcommand{\DHP}{\hardprobfont{DH}\xspace}
\newcommand{\DLin}{\hardprobfont{DLin}\xspace}
\newcommand{\XDH}{\hardprobfont{XDH}\xspace}
\newcommand{\CDH}{\hardprobfont{CDH}\xspace}
\newcommand{\LWE}{\hardprobfont{LWE}\xspace}
\newcommand{\SXDH}{\hardprobfont{SXDH}\xspace}
\newcommand{\DCR}{\hardprobfont{DCR}\xspace}
\newcommand{\dlog}{\ensuremath{\mathsf{dlog}}\xspace}

%--------------------------------------------------------
% Various
%--------------------------------------------------------

\newcommand{\map}{\ensuremath{\mathsf{map}}\xspace}
\newcommand{\mode}{\mathsf{mode}}
\newcommand{\token}{\ensuremath{\mathsf{token}}\xspace}
\newcommand{\seed}{\ensuremath{\mathsf{seed}}\xspace}
\newcommand{\inp}{\ensuremath{\mathsf{in}}\xspace}
\newcommand{\outp}{\ensuremath{\mathsf{out}}\xspace}
\newcommand{\circuit}{\ensuremath{\mathcal{C}}\xspace}
\newcommand{\size}[1]{\ensuremath{\left\vert #1 \right\vert}\xspace}
\newcommand{\myand}{\ensuremath{\mathsf{and}}\xspace}
\newcommand{\myxor}{\ensuremath{\mathsf{xor}}\xspace}

\newcommand{\mynot}{\ensuremath{\mathsf{not}}\xspace}
\newcommand{\Input}{\ensuremath{\mathsf{Input}}\xspace}


\newcommand{\hsha}{{\textrm{HMAC-SHA256}}}
\createpseudocodeblock {pcb}{center, boxed}{}{}{}

\newcommand{\samplePoly}{\pcalgostyle{Samp}}

\newcommand{\smallset}[1]{\{#1\}}
%\newcommand{\ro}{\mathcal{H}}

\DeclareRobustCommand{\michals}[2] {{\color{magenta}{$\big[$\scriptsize\textsf{Michal #1:}} #2$\big]$}}

\title{Insert Title}
\titlerunning{Insert Running Title}
\date{}

\ifnum\anonymous=0
\author{
	Coauthor Name\inst{1} \and
	Coauthor Name\inst{2}
}% Add author name here

\institute{Coauthor University\\
	\href{mailto:mail@mail.com}{mail@mail.com} \and
	Coauthor University\\
	\href{mailto:mail@mail.com}{mail@mail.com}
}  % Add institute here

\else
\author{} 
\institute{}
\fi

\begin{document}
	\title{Security Analysis of Modified Shamir's Secret Sharing Scheme}
	
	\author{Research Team}
	\maketitle
	
	\section{Problem Definition}
	In Mnemonics project we use a modified version of Shamir's Secret Sharing Scheme (SSS) \cite{Shamir79} which can be stated as below. 
	
	\begin{definition}\label{def:shamir}
		Let $\FF_q$ be a finite field with $q$ elements and $t,n \in \ZZ$. A $(t, n)$-thresholds secret sharing scheme shares a secret $s \in \FF_q$ among $n$ users such that any $t$ or more shares can reconstruct the secret but $t-1$ or less cannot. SSS is an ideal and perfect. A $(t, n)$-thresholds scheme can be defined in two phases as follows:
		
		\begin{enumerate}
			\item \textbf{Share Generation Phase:} 
			\begin{enumerate}
				\item A dealer chooses a secret polynomial $f (x)$ with degree $t - 1$.
				\begin{equation*}
				f (x) = a_{t-1}x^{t-1} + \ldots + a_1x + a_0 \in \FF_q [x]
				\end{equation*}
				where $a_0 = s$, and $a_i \in_{R} \FF_q$ for $i = 1, \ldots, t-1$. 
				\item The dealer sends the tuple $(x_{i}, f (x_{i}))$ as the share to $i$-th user, where $x_{i} \in \FF_q$ are all non-zero and distinct for $i = 1, \ldots, n$. 
			\end{enumerate}
			
			\item \textbf{Reconstruction Phase:} 
			If any $t$ or more users jointly perform a Lagrange interpolation with their shares, they obtain the secret polynomial $f (x)$, and $f(0)$ yields the secret $s$.
		\end{enumerate}
	\end{definition}
	
	In addition to the above definition, we also encode the digest of the secret as $f(-1)$ as stated in ``\textit{SLIP-0039 : Shamir's Secret-Sharing for Mnemonic Codes}''. Our modified scheme, e.i. Shamir's secret sharing with digest (SSSd), is as follows:
	\begin{enumerate}
		\item \textbf{Share Generation Phase:}  
		\begin{enumerate}
			\item Let $s \in \FF_q$ be the secret to be shared and $D \in \FF_q$ be it's digest such that 
			\begin{equation*}
			D = \hsha2(R,s)[:d] \concat R
			\end{equation*} 
			where $R$ is the randomness with length $m - d$ in bits for $m = \log_{2}{q}$ and some $d \in \ZZ$. In other words the digest share $D$ is composed of two parts; first $d$-bit part is the first $d$ bits of the output of the hash function and the remaining part is a randomness with $m - d$ bit length. Assume that we encode our secret as $s = f(0)$ and the digest $D = f(-1)$.
			\item Since we already have two points $(0, s)$ and $(-1, D)$, we sample random $t-2$ more points, i.e. $y_{1}, y_{2}, \ldots y_{t-2} \in_{R} \FF_q$.
			\item Using the above shares we perform Lagrange interpolation and construct the unique degree $t - 1$ polynomial $f (x)$ with respect to our points.  
			\item By evaluating $f (x)$ at the points $x_{t - 1}, \ldots x_{n}$ we generate $n - t + 2$ more shares.
			\item We output our shares as $(x_{i},y_{i})$ for $i = 1, \ldots n$.
		\end{enumerate}
		\item \textbf{Reconstruction Phase:}  
		\begin{enumerate}
			\item Recover the secret polynomial by applying Lagrange interpolation with $(x_{j}, y_{j})$ for $j = i_{1}, \ldots, i_{t}$.
			\item Compute the secret $s = f(0)$
			\item Compute the digest $D = f(-1)$
			\item If $\hsha(D[d:],s)[:d] = D[:d]$, then return $s$. Otherwise abort. 
		\end{enumerate}
	\end{enumerate}
	
	\section{Formal Proof}
	Firstly we show that the SSS is unconditionally secure even against unbounded adversaries.
	Let $\FF_{q}$ be a Galois field where $q = 2^{m}$ for some $m \in \ZZ$. 
	Let $\sP_{s}^{t - 1}$ be the set of all polynomials over $\FF_{q}$ with constant term $s$ and degree $t - 1$.
	The number of elements of $\sP_{s}^{t - 1}$ is exactly $q^{t-1}$. 
	We define an algorithm $\sdv$ which takes $n, t \in \ZZ$ and the secret $s$ as inputs, where $n$ is the number of shares and $t$ is the threshold. 
	The algoritm $\sdv(s, n, t)$ works as follows:
	
	\pcb{
		f(x) \sample \sP_{s}^{t - 1}\\
		\pcreturn f(x_1), \ldots, f(x_n)}
	for distinct $x_i \in \FF_q^*$ and $i = 1, \ldots, n$.
	
	Now we will show that for a fixed set of $t-1$ shares, the values that secret $s \in \FF_q$ can take are uniformly distributed in $\FF_q$. 
	To this end, we fix $t-1$ shares $y_1, \ldots, y_{t-1} \in \FF_q$. Let  $I \subset \{1, \ldots, n\}$ be the set of indices of the fixed $t-1$ shares. 
	
	There is exactly one polynomial $f(x) \in \sP_{s}^{t - 1}$ that satisfies $f(0) = s$ and $f(x_{i}) = y_{i}$ for $i \in I$.
	Therefore the probability that $\sdv(s, n, t)$ chooses $f(x)$ exactly $1/q^{t-1}$.
	Now assume that we have another secret $s'$ For the same fixed set of $t-1$ shares.  
	From the same discussion above we can say that there is exactly one polynomial $g(x) \in \sP_{s'}^{t - 1}$ that satisfies $g(0) = s'$ and $g(x_{i}) = y_{i}$ for every $i \in I$.
	Therefore, for distinct secrets $s,s' \in \FF_q$ we have exactly the same probabilities such that
	
	\begin{align*}
		\probsub{f(x) \sample {\sP_{s}^{t - 1}}}{f(x_1) = y_1, \ldots, f(x_{t-1}) = y_{t-1}} &= \probsub{g(x) \sample {\sP_{s'}^{t - 1}}}{g(x_1) = y_1, \ldots, g(x_{t-1}) = y_{t-1}} \\
		&= 1/q^{t-1}.
	\end{align*}
	
	In other words, for a fixed $t-1$ shares $y_1, \ldots, y_{t-1} \in \FF_q$, and a secret $s \in \FF_q$, there exist only one polynomial $f(x)$ with degree $t-1$, satisfying $f(x_i) = y_i$ for $i = 1, \ldots, t-1$ and $f(0) = s$.
	Since there exist a unique polynomial for every possible value of $s$ and the secret $s$ is an element of $\FF_q$, for every fixed $t-1$ shares, the secret can take exactly $|\FF_q|$ distinct values. 
	Namely, for a  degree $t - 1$ secret polynomial $f(x)$ and fixed $t-1$ points $(x_1, y_1), \ldots, (x_{t-1}, y_{t-1})$, such that $f(x_i) = y_i$, $x_i, y_i \in \FF_q$ and $x_i \neq 0$, all the possible values that the secret $s \in \FF_q$ can take, satisfying $f(0) = s$, are equiprobable, that is exactly $1/q$.  
	
	Therefore any adversary against the SSS who holds $t-1$ shares cannot do anything better than a random guess whose probability is exactly $1/q = 1/2^m$.
	
	Now we move on our modified scheme. 	
	In SSSd, the digest share, e.i. $D = k \concat R$, is also encoded as $f(-1)$, where  $k$ is the first $d$ bits of the output of $\hsha(R, s)$ and $R$ is an $m - d$ bit randomness. 
	\begin{definition} \label{def: q_d_e_security}
		A $\ppt$ algorithm is called $(Q, d, \epsilon)$-adversary for SSSd scheme if it makes at most $Q$ queries and breaks the SSSd scheme with probability greater than $Q \cdot 2^{d} \cdot \epsilon$, where $d$ is the length of the partial digest of the secret concetanated with a randomness and $\epsilon$ is the success probability of an unbounded adversary against the SSS. We say that SSSd is $(Q, d, \epsilon)$-secure if no $(Q, d, \epsilon)$-adversary exists. 
	\end{definition}
	
	\begin{theorem}
		SSSd is $(Q, d, \epsilon)$-secure in  the random oracle mode.
		More precisely, any $\ppt$ adversary against the SSSd succeeds with probability at most $Q \cdot 2^{d} \cdot \epsilon$, where $Q, d$ and $\epsilon$ are as defined in Definition \ref{def: q_d_e_security}.   
	\end{theorem}
	
	\begin{pf}
		Assume that we have a $\ppt$ adversary  $\bdv^{\hash(\cdot)}$ against the SSSd scheme which is given access to the random oracle $\hash(\cdot)$, takes the length of the partial digest $d$ and $t-1$ shares among $n$, i.e.~$y_i \in \FF_q$, $i \in I \subset \{1, \ldots, n\}$ where $I$ is the set of indexes with $|I| = t - 1$, as inputs and outputs the secret with probability at most $\epsilon'$.
		Let $\adv$ be another adversary against the SSS which is given to oracle access to $\bdv^{\hash(\cdot)}$ and holds $t-1$ shares, i.e. $\{y_i: i \in I\}$.
		On input $\{y_i: i \in I\}$, adversary $\adv$ runs as follows:
		\begin{enumerate}
			\item Guesses an oracle output $r_{0} \sample \{0,1\}^{\ell}$ such that the first $d$-bit part is the correct partial digest guess, where $\ell$ is the length of the output of the random oracle in bits.
			\item Runs $\bdv^{\hash(\cdot)}(d, \{y_i: i \in I\})$, 
			\item Guesses a query index $i \sample \{1, \ldots, Q\}$,
			\item Responds $\bdv^{\hash(\cdot)}$'s $j$-th query $M$ as follows:
			\begin{enumerate}
				\item If $j \neq i$, then responds with a random $r_{j} \sample \{0,1\}^{\ell}$, 
				\item If $j = i$, then responds with $r_{0}$.	
				\item Parses $M$ as $R'\concat s'$ and returns $s'$.
			\end{enumerate}
		\end{enumerate} 
		The success probability of the adversary $\adv$ depends on the following probabilities:
		\begin{itemize}
			\item The success probability of the adversary $\bdv^{\hash(\cdot)}$ which is at most $\epsilon'$,
			\item The probability of the correct index guess which is bounded with $\frac{1}{Q}$ and 
			\item The probability of the correct digest guess, i.e. $2^{-d}$.
		\end{itemize}
		Therefore, overall success probability of $\adv$ is 
		\begin{equation*}
		\condprob{s' = s}{s' \leftarrow \adv(\bdv^{\hash(\cdot)}, \{y_i: i \in I\})} \leq \epsilon' \cdot \frac{1}{Q} \cdot 2^{-d} \leq \epsilon,
		\end{equation*}
		which gives us $\epsilon' \leq Q \cdot 2^{d} \cdot \epsilon$. 
	\end{pf}

	\begin{rmrk}
		From the discussion about the security of the SSS scheme above, we know that probability of a random guess is $2^{-m}$, which is the best probability that any unbounded adversary succeeds. Therefore we have $\epsilon' \leq Q \cdot 2^{-m+d}$. 
	\end{rmrk}
	%\section{Introduction}
	%\label{sec:introduction}
	%\input{tex_files/01_introduction}
	
	\bibliographystyle{alpha}
	\bibliography{cryptobib/abbrev3,cryptobib/crypto}
	
\end{document} 