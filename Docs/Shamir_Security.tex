%
% Parameters
%

\ifdefined\fullversion
\else
\def\fullversion{1}    % 0 = conference version; 1 = full version
\fi

\ifdefined\cameraversion
\else
\def\cameraversion{0}    % 0 = long version; 1 = proceedings version
\fi

\def\showoverflow{1}   % 1 = show overflows
\def\allow{1}      % 0 = remove todo command
\def\anonymous{1}      % 1 = anonymous

%
% Document class 
%

\documentclass[envcountsame,runningheads,notitlepage]{llncs}


\ifnum\fullversion=1
\usepackage[a4paper, margin=1.5in]{geometry}
\setlength{\marginparwidth}{2.5cm}
\fi



%
% Custom header
%

%!TEX root=../main.tex

% Packages
%

\usepackage[utf8]{inputenc}
\usepackage[T1]{fontenc}
\usepackage{hyperref}
\usepackage{verbatim}
\usepackage{tikz}
\usetikzlibrary{positioning,calc}
\usepackage{xspace}
\usepackage{mathtools}
\usepackage{pifont}
\usepackage{etoolbox}
\usepackage[normalem]{ulem}
\usepackage{booktabs}
\usepackage{array}
\usepackage[capitalise,noabbrev]{cleveref}
\usepackage{cite}
\usepackage{multibib}
\usepackage{url}
\usepackage{algorithm}
\usepackage{algpseudocode}
\usepackage{paralist}
\usepackage{mathrsfs}
\usepackage{relsize}
\usepackage{stmaryrd}
\usepackage[textsize=small]{todonotes}
\usepackage{multirow}

\DeclareFontFamily{U}{mathx}{\hyphenchar\font45}
\DeclareFontShape{U}{mathx}{m}{n}{<-> mathx10}{}
\DeclareSymbolFont{mathx}{U}{mathx}{m}{n}
\DeclareMathAccent{\widebar}{0}{mathx}{"73}

%\let\vec\vec
\usepackage{amssymb,amsmath}
\let\vec\boldsymbol


\usepackage[
n, % or lambda
advantage,
operators,
sets,
adversary,
landau,
probability,
notions,
logic,
ff,
mm,
primitives,
events,
complexity,
oracles,
asymptotics,
keys
]{cryptocode}

\newtoggle{notes}
\toggletrue{notes} % set to false to remove colored notes from the paper

%--------------------------------------------------------
% Editorial
%--------------------------------------------------------

\newcommand{\redunderline}[1]{\textcolor{red}{\underline{\textcolor{black}{#1}}}} 
\newcommand{\TBW}{\textcolor{blue}{\textbf{To Be Written...}}}
\newcommand{\Note}[1]{\textcolor{magenta}{ $\langle \! \langle$ #1 $\rangle \! \rangle$}}
\newcommand{\todonote}[1]{\todo[inline]{MyName: #1}}

\DeclareRobustCommand{\ahmets}[2]  {{\color{blue} $[$ \textsf{Ahmet #1:} #2 $]$ }}
\DeclareRobustCommand{\michals}[2]  {{\color{magenta} $[$ \textsf{Michal #1:} #2 $]$ }}

%--------------------------------------------------------
% General notations
%--------------------------------------------------------


\usepackage{xspace}
\newcommand{\ie}{i.e.\@\xspace}
\newcommand{\sA}{{\mathcal A}}
\newcommand{\sB}{{\mathcal B}}
\newcommand{\sC}{{\mathcal C}}
\newcommand{\sD}{{\mathcal D}}
\newcommand{\sE}{{\mathcal E}}
\newcommand{\sF}{{\mathcal F}}
\newcommand{\sG}{{\mathcal G}}
\newcommand{\sH}{{\mathcal H}}
\newcommand{\sI}{{\mathcal I}}
\newcommand{\sJ}{{\mathcal J}}
\newcommand{\sK}{{\mathcal K}}
\newcommand{\sL}{{\mathcal L}}
\newcommand{\sM}{{\mathcal M}}
\newcommand{\sN}{{\mathcal N}}
\newcommand{\sO}{{\mathcal O}}
\newcommand{\sP}{{\mathcal P}}
\newcommand{\sQ}{{\mathcal Q}}
\newcommand{\sR}{{\mathcal R}}
\newcommand{\sS}{{\mathcal S}}
\newcommand{\sT}{{\mathcal T}}
\newcommand{\sU}{{\mathcal U}}
\newcommand{\sV}{{\mathcal V}}
\newcommand{\sW}{{\mathcal W}}
\newcommand{\sX}{{\mathcal X}}
\newcommand{\sY}{{\mathcal Y}}
\newcommand{\sZ}{{\mathcal Z}}

% Sonderbuchstaben mit Doppellinie
\renewcommand{\pcsetstyle}[1]{\ensuremath{\mathbb{#1}}}

\renewcommand{\probname}{Pr}
\renewcommand{\pcadvstyle}[1]{\ensuremath{\mathcal{#1}}}

\renewcommand{\pcadvantagename}{\mathsf{Adv}}
\renewcommand{\pcadvantagesuperstyle}[1]{\mathrm{\MakeLowercase{#1}}}
\renewcommand{\pcadvantagesubstyle}[1]{#1}

\renewcommand{\pckeystyle}[1]{\ensuremath{\mathsf{#1}}}


%--------------------------------------------------------
% Standard proba, games, proofs, sampling, distributions
%--------------------------------------------------------

\newcommand{\Good}{\ensuremath{\mathsf{Good}}\xspace}
\newcommand{\Bad}{\ensuremath{\mathsf{Bad}}\xspace}
\newcommand{\equivStat}{\ensuremath{\overset{\mathsf{stat}}{\equiv}}\xspace}
\newcommand{\equivComp}{\ensuremath{\overset{\mathsf{comp}}{\equiv}}\xspace}
\newcommand{\view}{\ensuremath{\textsc{View}}\xspace}

\newcommand{\Hyb}{\ensuremath{\mathsf{Hyb}}\xspace}
\newcommand{\Exp}{\ensuremath{\mathsf{Exp}}\xspace}
\newcommand{\myGame}{\ensuremath{\mathsf{Game}}\xspace}
\newcommand{\Event}{\ensuremath{\mathsf{E}}\xspace}
\newcommand{\Span}{\ensuremath{\mathsf{Span}}}

\newcommand{\probb}[2]{{\Pr}_{#1}\left[\,{#2}\,\right]}
\newcommand{\Dx}{\mathcal{D}}
\newcommand{\Hx}{\mathcal{H}}
\newcommand{\Sx}{\mathcal{S}}
\newcommand{\Lx}{\mathcal{L}}
\newcommand{\Dist}{\mathcal{D}}
\newcommand{\Expect}{\ensuremath{\mathbb{E}}\xspace}
\newcommand{\Sample}{\ensuremath{\mathsf{Sample}}\xspace}
\newcommand{\Sim}{\ensuremath{\mathsf{Sim}}\xspace}


%--------------------------------------------------------
% Hard problems
%--------------------------------------------------------

\newcommand{\hardprobfont}[1]{\texorpdfstring{\ensuremath{\textsf{#1}}}{#1}}
\newcommand{\kLIN}{\ensuremath{k\text{-}\mathsf{Lin}}\xspace}
\newcommand{\SEDL}{\hardprobfont{SEDL}\xspace}
\newcommand{\DL}{\hardprobfont{DL}\xspace}
\newcommand{\DDH}{\hardprobfont{DDH}\xspace}
\newcommand{\kerDH}{\hardprobfont{kerDH}\xspace}
\newcommand{\kernel}{\mathsf{ker}\xspace}
\newcommand{\DHP}{\hardprobfont{DH}\xspace}
\newcommand{\DLin}{\hardprobfont{DLin}\xspace}
\newcommand{\XDH}{\hardprobfont{XDH}\xspace}
\newcommand{\CDH}{\hardprobfont{CDH}\xspace}
\newcommand{\LWE}{\hardprobfont{LWE}\xspace}
\newcommand{\SXDH}{\hardprobfont{SXDH}\xspace}
\newcommand{\DCR}{\hardprobfont{DCR}\xspace}
\newcommand{\dlog}{\ensuremath{\mathsf{dlog}}\xspace}

%--------------------------------------------------------
% Various
%--------------------------------------------------------

\newcommand{\map}{\ensuremath{\mathsf{map}}\xspace}
\newcommand{\mode}{\mathsf{mode}}
\newcommand{\token}{\ensuremath{\mathsf{token}}\xspace}
\newcommand{\seed}{\ensuremath{\mathsf{seed}}\xspace}
\newcommand{\inp}{\ensuremath{\mathsf{in}}\xspace}
\newcommand{\outp}{\ensuremath{\mathsf{out}}\xspace}
\newcommand{\circuit}{\ensuremath{\mathcal{C}}\xspace}
\newcommand{\size}[1]{\ensuremath{\left\vert #1 \right\vert}\xspace}
\newcommand{\myand}{\ensuremath{\mathsf{and}}\xspace}
\newcommand{\myxor}{\ensuremath{\mathsf{xor}}\xspace}

\newcommand{\mynot}{\ensuremath{\mathsf{not}}\xspace}
\newcommand{\Input}{\ensuremath{\mathsf{Input}}\xspace}


\newcommand{\hsha}{{\textrm{HMAC-SHA256}}}
\createpseudocodeblock {pcb}{center, boxed}{}{}{}

\newcommand{\samplePoly}{\pcalgostyle{Samp}}

\newcommand{\smallset}[1]{\{#1\}}
%\newcommand{\ro}{\mathcal{H}}

\title{Insert Title}
\titlerunning{Insert Running Title}
\date{}

\ifnum\anonymous=0
\author{
  Coauthor Name\inst{1} \and
  Coauthor Name\inst{2}
}% Add author name here

\institute{Coauthor University\\
  \href{mailto:mail@mail.com}{mail@mail.com} \and
  Coauthor University\\
  \href{mailto:mail@mail.com}{mail@mail.com}
}  % Add institute here

\else
\author{} 
\institute{}
\fi

\begin{document}
	\title{Security Analysis of Modified Shamir's Secret Sharing Scheme}
	
	\author{Research Team}
	\maketitle
	
	\section{Problem Definition}
	In Mnemonics project we use a modified version of Shamir's Secret Sharing Scheme (SSS) \cite{shamirSSS} which can be stated as below. 
	
	\begin{definition}\label{def:shamir}
		 Let $\FF_q$ be a finite field with $q$ elements and $t,n \in \ZZ$. A $(t, n)$-thresholds secret sharing scheme shares a non-zero secret $s \in \FF_q$ among $n$ users such that any $t$ or more shares can reconstruct the secret but $t-1$ or less cannot. Shamir's secret sharing scheme is an ideal and perfect A $(t, n)$-thresholds scheme and can be defined in two phases as follows:
		 
		\begin{enumerate}
			\item \textbf{Share Generation Phase:} 
				\begin{enumerate}
					\item A dealer chooses a secret polynomial $f (x)$ with degree $t - 1$.
						\begin{equation*}
						f (x) = a_{t-1}x^{t-1} + \ldots + a_1x + a_0 \in \FF_q [x]
						\end{equation*}
					where $a_0 = s$, and $a_i \in_{R} \FF_q$ for $i = 1, \ldots, t-1$. 
					\item The dealer sends the tuple $(x_{i}, f (x_{i}))$ as the share to $i$-th user, where $x_{i} \in \FF_q$ are all non-zero and distinct for $i = 1, \ldots, n$. 
				\end{enumerate}
		
			\item \textbf{Reconstruction Phase:} 
			If any $t$ or more users jointly perform a Lagrange interpolation with their shares, they obtain the secret polynomial $f (x)$, and $f(0)$ yields the secret $s$.
		\end{enumerate}
	\end{definition}

	In addition to the above definition, we also encode the digest of the secret as $f(-1)$ as stated in ``\textit{SLIP-0039 : Shamir's Secret-Sharing for Mnemonic Codes}''. Our modified scheme is as follows:
	\begin{enumerate}
		\item \textbf{Share Generation Phase:}  
			\begin{enumerate}
				\item Let $s \in \FF_q$ be the secret to be shared and $D \in \FF_q$ be it's digest such that 
				\begin{equation*}
				D = \textrm{HMAC-SHA256}(R,s)[:4] || R
				\end{equation*} 
				where $R$ is the randomness with length $m-32$ in bits for $m = \log_{2}{q}$. In other words the digest share $D$ is conposed of two parts; first 32-bit part is the first 32 bits of the output of the hash function and the remaining part is a randomness with $m-32$ bit length. Assume that we encode our secret as $s = f(0)$ and the digest $D = f(-1)$.
				\item Since we already have two points $(0, s)$ and $(-1, D)$, we sample random $t-2$ more points, i.e. $y_{1}, y_{2}, \ldots y_{t-2} \in_{R} \FF_q$.
				\item Using the above shares we perform Lagrange interpolation and construct the unique degree $t-1$ polynomial $f (x)$ with respect to our points.  
				\item By evaluating $f (x)$ at the points $x_{t-1}, \ldots x_{n}$ we generate $n - t + 2$ more shares.
				\item We output our shares as $(x_{i},y_{i})$ for $i = 1, \ldots n$.
			\end{enumerate}
		\item \textbf{Reconstruction Phase:}  
			\begin{enumerate}
				\item Recover the secret polynomial by applying Lagrange interpolation with $(x_{j},y_{j})$ for $j = i_{1}, \ldots, i_{t}$.
				\item Compute the secret $s = f(0)$
				\item Compute the digest $D = f(-1)$
				\item If $\textrm{HMAC-SHA256}(D[4:],s)[:4] = D[:4]$, then return $s$. Otherwise abort. 
			\end{enumerate}
	\end{enumerate}
	
	\section{Security Analysis}
	Let $\adv$ be a PPT adversary holding $t-1$ shares among $n$. WLOG we can assume that $\adv$ has $(x_{i}, y_{i})$ for $i = 1, \ldots, t-1$. Lagrange interpolation formula tells us that constructing a degree $t-1$ polynomial requires at least $t$ points. So $\adv$ has a missing point and let assume that this missing point be $(x_{t},y_{t}) = (-1, D)$.
	
	The Lagrange interpolation formula is
		\begin{equation}\label{eq:Lag_int}
		f(x) = \sum\limits_{j = 1}^{t} y_{j}\ell_{j}(x)
		\end{equation}
	, where $\ell_{j}(x)$ is the  Lagrange basis polynomial WRT the $j$-th point, i.e.
		\begin{equation}\label{eq:Lag_Basis}
		\ell_{j}(x) = \prod\limits_{\substack{k = 1 \\ k \neq j}}^{t} \frac{x - x_{k}}{x_{j} - x_{k}}
		\end{equation}
	
	Since $\adv$ knows the $x$ coordinates of $t$ points she holds, she can compute the Lagrange basis polynomials according to (\ref{eq:Lag_Basis}). 
	
	Now, using (\ref{eq:Lag_int}) she writes
	
	\begin{align*}
		s &= f(0)\\
		  &= \sum\limits_{j = 1}^{t-1} y_{j}\ell_{j}(0) + y_{t}\ell_{t}(0)
	\end{align*}
	and 
	\begin{equation}\label{eq:thelast}
		s - \sum\limits_{j = 1}^{t-1} y_{j}\ell_{j}(0) = y_{t}\ell_{t}(0)
	\end{equation}
	
	We know from the Definition \ref{def:shamir} that the secret and the shares are non-zero. Since the $x$ coordinates are all distinct, the Lagrange basis polynomials are also non-zero. There exist one unknown in both sides of (\ref{eq:thelast}), i.e. $s$ and $D$. Since $s,D \in \FF_q^{*}$, for each possible value of $s \in \FF_q$ there exist a unique value $y_{t} \in \FF_q$. But Since $y_{t} = D$, $\adv$ can perform a brute-force attack on the secret $s$ and using (\ref{eq:thelast}) she eliminates the candidate secret giving invalid digest. More precisely $\adv$ can proceed as follows.
		\begin{itemize}
			\item Picks an $s'$ from $\FF_q$,
			\item set $D' = \frac{s' - \sum\limits_{j = 1}^{t-1} y_{j}\ell_{j}(0)}{\ell_{t}(0)}$,
			\item Checks wether the first 4 byte of $D'$ is equal to $\textrm{HMAC-SHA256}(R, s')[:4]$, where $R$ is the remaining part of $D'$.
			\item If they are equal, then $s' = s$ with probability $1/2^{m-32}$. 
			\item Otherwise eliminate the candidate secret $s'$ and try with a fresh $s' \in \FF_q$. 
		\end{itemize}
 Since above attack works for $2^{32}$ different values of $s' \in \FF_q$, the brute-force space decreases from $\sO(2^{m})$ to $\sO(2^{m-32})$. 
	
	to be continued...
	
%\section{Introduction}
%\label{sec:introduction}
%\input{tex_files/01_introduction}

\ifnum\fullversion=1
  \bibliographystyle{splncs03}
 \else
   \bibliographystyle{alpha-short}
 \fi
\bibliography{cryptobib/abbrev3,cryptobib/crypto,add}

\end{document} 